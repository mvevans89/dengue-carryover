\documentclass[12pt]{article}
\usepackage[paper=letterpaper, margin=0.8in]{geometry}

%% The graphicx package provides the includegraphics command.
\usepackage{graphicx}
%% The amssymb package provides various useful mathematical symbols
\usepackage{amssymb}
%% The amsthm package provides extended theorem environments
%% \usepackage{amsthm}
%% fix strange gensymb error
\usepackage{textcomp}
%% symbols, especially degree
\usepackage{gensymb}
%% scientific units
\usepackage{siunitx}
%% line spacing
\usepackage{setspace}
\doublespacing
%% put figures at the end
%\usepackage[nomarkers]{endfloat}
%% allow hyperlinks
%\usepackage{hyperref}
%% color comments
\usepackage{soul}
\usepackage{color}
%% left justification in tables
\usepackage{array}
\newcolumntype{P}[1]{>{\raggedright\arraybackslash}p{#1}}
%%references
\usepackage[round]{natbib}
%%landscape orientation
\usepackage{pdflscape}

\begin{document}

{\Large
\textbf\newline{Carry-over effects of larval microclimate on the transmission potential of a mosquito-borne pathogen}}

\bigskip

Michelle V. Evans,
Justine C. Shiau,
Nicole Solano,
Melinda A. Brindley,
John M. Drake,
Courtney C. Murdock
\smallskip

{\Large{Supplemental Information}}

\subsection{Intrinsic growth rates (r')}

We calculated the per capita population growth rate per following Livdahl and Sugihara \citep{livdahl1984} {Eq. \ref{eq:1}}:

\begin{equation} \label{eq:1}
r' = \frac{ln(\frac{1}{N_0}\sum_{x}^{ }{A_x}f(\bar{w_x}))}{D+\frac{\sum_{x}^{ }xA_xf(\bar{w_x})}{\sum_{x}^{ }A_xf(\bar{w_x})}}
\end{equation}

Where $N_0$ is the initial number of female mosquitoes (assumed to be 50\% of the larvae, n=50), $A_x$ is the number of mosquitoes emerging on day $x$, $D$ is the time to reproduction following emergence (assumed to be 14 days \citep{livdahl1991}), and $f(\bar{w_x})$ is fecundity as a function of mean wing size on day $x$ ($w_x$; Equation \ref{eq:2}). This relationship is assumed to be linear and calculated via Lounibos et al. \citep{lounibos2002}:

\begin{equation} \label{eq:2}
f(\bar{w_x}) = -121.240 + (78.02 \times \bar{w_x})
\end{equation}

\subsection{Vectorial Capacity}

We calculated the vectorial capacity ($VC$; Equation \ref{eq:3}) for each site and season using a temperature-dependent mechanistic dengue model defined in Mordecai et al. \citep{mordecai2017}.

\begin{equation} \label{eq:3}
VC(T) =\frac{a(T)^2b(T)c(T)e^{-\mu (T)/EIR(T)} EFD(T) p_{EA}(T) MDR(T)} {\mu(T)^2}
\end{equation}

Here, mosquito traits are a function of temperature, $T$, as described in Table \ref{table:traits}:

\begin{table}[h]
\centering
\begin{tabular}{P{0.15\linewidth}P{0.3\linewidth}P{0.25\linewidth}P{0.25\linewidth}}
\hline
\textbf{Parameter} & \textbf{Definition} & \textbf{Without carry-over effects} & \textbf{With carry-over effects}\\
\hline
$a(T)$ & Per-mosquito bite rate & Mordecai et al. 2017 & Mordecai et al. 2017 \\
$b(T)c(T)$* & Vector competence & Mordecai et al. 2017 & Current Study \\
$\mu(T)$ & Adult mosquito mortality rate & Mordecai et al. 2017 & Mordecai et al. 2017 \\
$EIR(T)$ & Extrinsic incubation rate (inverse of extrinsic incubation period) & Mordecai et al. 2017 & Mordecai et al. 2017 \\
$EFD(T)$* & Number of eggs produced per female mosquito per day & Mordecai et al. 2017 & Current Study \\
$p_{EA}(T)$ & Egg-to-adult survival probability & Current Study & Current Study \\
$MDR(T)$ & Mosquito immature development rate & Current Study & Current Study \\
\hline
\end{tabular}
\caption{Sources of parameters used in the $VC$ equation. Parameters sourced from \citep{mordecai2017} were mathematically estimated at a constant temperature of 27 \degree C. Parameters that included carry-over effects are starred.}
\label{table:traits}
\end{table}

Site-level $VC$ was calculated using a combination of traits empirically measured in this study and traits estimated from thermal response models as described in \citep{mordecai2017}. The bite rate ($a(T)$), adult mosquito mortality rate ($\mu(T)$), and extrinsic incubation rate ($EIR(T)$), were calculated for mosquitoes at a constant 27 \degree C using temperature dependent functions from \citep{mordecai2017}. Vector competence ($b(T)c(T)$) was calculated as the proportion of infectious mosquitoes per site as found by our dengue infection assays. The number of eggs produced per female per day ($EFD(T)$) was calculated by estimating fecundity from average female wing length following Eq. \ref{eq:2}, and then dividing this by the expected lifespan of mosquitoes ($1/\mu$). The egg-to-adult survival probability ($p_{EA}(T)$) was defined as the average proportion of adults emerging at a site. The mosquito immature development rate ($MDR(T)$) was calculated as the inverse of the mean time to emergence for female mosquitoes per site, resulting in a daily rate of development.

\newpage

\bibliographystyle{vancouver}
\bibliography{references}

\end{document}
