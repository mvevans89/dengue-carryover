%% BioMed_Central_Tex_Template_v1.06
%%                                      %
%  bmc_article.tex            ver: 1.06 %
%                                       %

%%IMPORTANT: do not delete the first line of this template
%%It must be present to enable the BMC Submission system to
%%recognise this template!!

%%%%%%%%%%%%%%%%%%%%%%%%%%%%%%%%%%%%%%%%%
%%                                     %%
%%  LaTeX template for BioMed Central  %%
%%     journal article submissions     %%
%%                                     %%
%%          <8 June 2012>              %%
%%                                     %%
%%                                     %%
%%%%%%%%%%%%%%%%%%%%%%%%%%%%%%%%%%%%%%%%%


%%%%%%%%%%%%%%%%%%%%%%%%%%%%%%%%%%%%%%%%%%%%%%%%%%%%%%%%%%%%%%%%%%%%%
%%                                                                 %%
%% For instructions on how to fill out this Tex template           %%
%% document please refer to Readme.html and the instructions for   %%
%% authors page on the biomed central website                      %%
%% http://www.biomedcentral.com/info/authors/                      %%
%%                                                                 %%
%% Please do not use \input{...} to include other tex files.       %%
%% Submit your LaTeX manuscript as one .tex document.              %%
%%                                                                 %%
%% All additional figures and files should be attached             %%
%% separately and not embedded in the \TeX\ document itself.       %%
%%                                                                 %%
%% BioMed Central currently use the MikTex distribution of         %%
%% TeX for Windows) of TeX and LaTeX.  This is available from      %%
%% http://www.miktex.org                                           %%
%%                                                                 %%
%%%%%%%%%%%%%%%%%%%%%%%%%%%%%%%%%%%%%%%%%%%%%%%%%%%%%%%%%%%%%%%%%%%%%

%%% additional documentclass options:
%  [doublespacing]
%  [linenumbers]   - put the line numbers on margins

%%% loading packages, author definitions

%\documentclass[twocolumn]{bmcart}% uncomment this for twocolumn layout and comment line below
\documentclass{bmcart}

%%% Load packages
%\usepackage{amsthm,amsmath}
%\RequirePackage{natbib}
%\RequirePackage[authoryear]{natbib}% uncomment this for author-year bibliography
%\RequirePackage{hyperref}
\usepackage[utf8]{inputenc} %unicode support
%\usepackage[applemac]{inputenc} %applemac support if unicode package fails
%\usepackage[latin1]{inputenc} %UNIX support if unicode package fails


%%%%%%%%%%%%%%%%%%%%%%%%%%%%%%%%%%%%%%%%%%%%%%%%%
%%                                             %%
%%  If you wish to display your graphics for   %%
%%  your own use using includegraphic or       %%
%%  includegraphics, then comment out the      %%
%%  following two lines of code.               %%
%%  NB: These line *must* be included when     %%
%%  submitting to BMC.                         %%
%%  All figure files must be submitted as      %%
%%  separate graphics through the BMC          %%
%%  submission process, not included in the    %%
%%  submitted article.                         %%
%%                                             %%
%%%%%%%%%%%%%%%%%%%%%%%%%%%%%%%%%%%%%%%%%%%%%%%%%


\def\includegraphic{}
\def\includegraphics{}



%%% Put your definitions there:
\startlocaldefs
\endlocaldefs


%%% Begin ...
\begin{document}

%%% Start of article front matter
\begin{frontmatter}

\begin{fmbox}
\dochead{Research}

%%%%%%%%%%%%%%%%%%%%%%%%%%%%%%%%%%%%%%%%%%%%%%
%%                                          %%
%% Enter the title of your article here     %%
%%                                          %%
%%%%%%%%%%%%%%%%%%%%%%%%%%%%%%%%%%%%%%%%%%%%%%

\title{Carry-over effects of larval microclimate on the transmission potential of a mosquito-borne pathogen}

%%%%%%%%%%%%%%%%%%%%%%%%%%%%%%%%%%%%%%%%%%%%%%
%%                                          %%
%% Enter the authors here                   %%
%%                                          %%
%% Specify information, if available,       %%
%% in the form:                             %%
%%   <key>={<id1>,<id2>}                    %%
%%   <key>=                                 %%
%% Comment or delete the keys which are     %%
%% not used. Repeat \author command as much %%
%% as required.                             %%
%%                                          %%
%%%%%%%%%%%%%%%%%%%%%%%%%%%%%%%%%%%%%%%%%%%%%%

\author[
   addressref={aff1},                   % id's of addresses, e.g. {aff1,aff2}
   corref={aff1},                       % id of corresponding address, if any
   noteref={n1},                        % id's of article notes, if any
   email={jane.e.doe@cambridge.co.uk}   % email address
]{\inits{JE}\fnm{Jane E} \snm{Doe}}
\author[
   addressref={aff1,aff2},
   email={john.RS.Smith@cambridge.co.uk}
]{\inits{JRS}\fnm{John RS} \snm{Smith}}

%%%%%%%%%%%%%%%%%%%%%%%%%%%%%%%%%%%%%%%%%%%%%%
%%                                          %%
%% Enter the authors' addresses here        %%
%%                                          %%
%% Repeat \address commands as much as      %%
%% required.                                %%
%%                                          %%
%%%%%%%%%%%%%%%%%%%%%%%%%%%%%%%%%%%%%%%%%%%%%%

\address[id=aff1]{%                           % unique id
  \orgname{Department of Zoology, Cambridge}, % university, etc
  \street{Waterloo Road},                     %
  %\postcode{}                                % post or zip code
  \city{London},                              % city
  \cny{UK}                                    % country
}
\address[id=aff2]{%
  \orgname{Marine Ecology Department, Institute of Marine Sciences Kiel},
  \street{D\"{u}sternbrooker Weg 20},
  \postcode{24105}
  \city{Kiel},
  \cny{Germany}
}

%%%%%%%%%%%%%%%%%%%%%%%%%%%%%%%%%%%%%%%%%%%%%%
%%                                          %%
%% Enter short notes here                   %%
%%                                          %%
%% Short notes will be after addresses      %%
%% on first page.                           %%
%%                                          %%
%%%%%%%%%%%%%%%%%%%%%%%%%%%%%%%%%%%%%%%%%%%%%%

\begin{artnotes}
%\note{Sample of title note}     % note to the article
\note[id=n1]{Equal contributor} % note, connected to author
\end{artnotes}

\end{fmbox}% comment this for two column layout

%%%%%%%%%%%%%%%%%%%%%%%%%%%%%%%%%%%%%%%%%%%%%%
%%                                          %%
%% The Abstract begins here                 %%
%%                                          %%
%% Please refer to the Instructions for     %%
%% authors on http://www.biomedcentral.com  %%
%% and include the section headings         %%
%% accordingly for your article type.       %%
%%                                          %%
%%%%%%%%%%%%%%%%%%%%%%%%%%%%%%%%%%%%%%%%%%%%%%

\begin{abstractbox}

\begin{abstract} % abstract
\parttitle{First part title} %if any
Text for this section.

\parttitle{Second part title} %if any
Text for this section.
\end{abstract}

%%%%%%%%%%%%%%%%%%%%%%%%%%%%%%%%%%%%%%%%%%%%%%
%%                                          %%
%% The keywords begin here                  %%
%%                                          %%
%% Put each keyword in separate \kwd{}.     %%
%%                                          %%
%%%%%%%%%%%%%%%%%%%%%%%%%%%%%%%%%%%%%%%%%%%%%%

\begin{keyword}
\kwd{sample}
\kwd{article}
\kwd{author}
\end{keyword}

% MSC classifications codes, if any
%\begin{keyword}[class=AMS]
%\kwd[Primary ]{}
%\kwd{}
%\kwd[; secondary ]{}
%\end{keyword}

\end{abstractbox}
%
%\end{fmbox}% uncomment this for twcolumn layout

\end{frontmatter}

%%%%%%%%%%%%%%%%%%%%%%%%%%%%%%%%%%%%%%%%%%%%%%
%%                                          %%
%% The Main Body begins here                %%
%%                                          %%
%% Please refer to the instructions for     %%
%% authors on:                              %%
%% http://www.biomedcentral.com/info/authors%%
%% and include the section headings         %%
%% accordingly for your article type.       %%
%%                                          %%
%% See the Results and Discussion section   %%
%% for details on how to create sub-sections%%
%%                                          %%
%% use \cite{...} to cite references        %%
%%  \cite{koon} and                         %%
%%  \cite{oreg,khar,zvai,xjon,schn,pond}    %%
%%  \nocite{smith,marg,hunn,advi,koha,mouse}%%
%%                                          %%
%%%%%%%%%%%%%%%%%%%%%%%%%%%%%%%%%%%%%%%%%%%%%%

%%%%%%%%%%%%%%%%%%%%%%%%% start of article main body
% <put your article body there>

%%%%%%%%%%%%%%%%
%% Background %%
%%
\section*{Introduction}

Climate plays an important role in the transmission of mosquito-borne pathogens, determining the geographic range of disease vectors and shaping transmission dynamics.
Heterogeneity in environmental conditions can directly shape individual-level variation in character traits relevant to mosquito population dynamics \citep{delatte2009}, as well as pathogen transmission \citep{murdock2012}.
However, in addition to these direct effects, mosquito phenotypes can be shaped indirectly by the environmental conditions experienced in previous life history stages, a phenomenon known as carry-over effects \citep{harrison2011}.
Carry-over effects have been documented in a wide-range of species with complex life cycles, such as amphibians \citep{vonesh2005}, migratory birds \citep{norris2006}, and insects \citep{deblock2005a}.
Similarly, the mosquito life cycle is characterized by ontogenetic niche shifts, with a larval aquatic stage and an adult terrestrial stage.
Following these studies, we reason that the thermal environment a mosquito experiences during its larval stage is likely to have lasting impacts on adult traits, and, ultimately, on transmission potential.

Although it has been previously demonstrated that larval environmental temperature can alter mosquito traits important for transmission, the net effect of temperature-mediated carry-over effects on overall transmission potential is ambiguous.
Current models of mosquito-borne disease typically incorporate direct effects of temperature, despite evidence that carry-over effects can have large impacts on adult phenotypes \citep{muturi2011c, muturi2011a, price2015}.
Additionally, laboratory studies designed to estimate temperature-mediated carry-over effects are often conducted across a wider range of temperatures than mosquitoes experience in the field \citep{cator2013}, which are not easily ``scaled-up'' to explain transmission across a landscape when incorporated into temperature-dependent models of mosquito-borne disease \citep{reiner2013}.

We hypothesize that relevant environmental variation during the larval stage will have lasting impacts on adult traits that are relevant for mosquito population dynamics and pathogen transmission.
To assess the implications of omitting carry-over effects, we used data collected from a semi-field experiment in a the \textit{Aedes albopictus}-dengue virus (DENV) system to parameterize a mechanistic transmission model.
We then compared model predictions when carry-over effects were incorporated relative to when they were excluded.

\section*{Methods}

To explore the effects of microclimate variation across an urban landscape, we used an impervious surface map (National Land Cover Database 2011 \citep{xian2011} to select three replicate sites ($30m \times 30m$) each of low (0-5\%), intermediate (6-40\%), and high (41-100\%) impervious surface. Percent impervious surface is an accurate predictor of land surface temperature, particularly for urban landscapes \citep{yuan2007}, and allowed us to ensure our sites exhibited the full range of urban microclimates. To select our sites, we calculated the percent impervious surface of each $30m \times 30m$ pixel using a moving focal window of $210m \times 210m$, as the surrounding impervious surface can affect the microclimate in the pixel of interest. We then classified each pixel based on the mean impervious surface within its focal window, with 0 - 5 \% representing low, 6 - 40 \% representing intermediate, and 41 - 100\% representing high. Because impervious surface is an effective classifier of urban land classes \citep{lu2006}, we identified the sites as rural, suburban, and urban with low, intermediate, and high impervious surface scores, respectively. Final site selection was constrained by access and permissions, however the final distribution of sites was chosen to ensure all sites were at least 2 miles from others of the same land class, and were evenly distributed across the study area (Fig. \ref{Fig:siteMap}).

To capture natural microclimate variation mosquitoes experience in the field, we chose three replicate sites (30m\textsuperscript{2}) each of low (0-5\%), intermediate (6-40\%), and high (41-100\%) impervious surface, representing rural, suburban, and urban land classes, respectively, that were interspersed across Athens-Clarke County, GA following methods outlined in Murdock et al. \citep{murdock2017} (Fig.\ref{Fig:map}).
Within each site, we evenly distributed four plastic trays, each containing 100 first instar \textit{Ae. albopictus} larvae and 1L of leaf infusion.
Leaf infusion was prepared as described in Murdock et al. \citep{murdock2017}.
Trays were screened with a fine mesh, placed in a wire cage to deter wildlife, covered with a clear plastic vinyl to keep rainwater from entering, and were placed in full shade.
We added deionized water to trays after two weeks to prevent trays from drying up and to maintain a total water volume at 1L.
We placed data loggers (Monarch Instruments: RFID Temperature Track-It logger) in vegetation next to each tray, approximately 3 feet above the ground.
Data loggers recorded instantaneous temperature and relative humidity at ten minute intervals throughout the study period.
Sites were visited daily from Aug. 1 to Sept. 3, 2016 (summer replicate) and Sept. 26 to Nov. 8, 2016 (fall replicate) to quantify the number of male and female mosquitoes emerging by tray per day, mosquito body size, and the proportion of mosquitoes that can transmit DENV (vector competence).

\subsection*{Dengue virus \textit{in vitro} culturing and mosquito infections}

We propagated DENV-2 virus stock (PRS 225 488) by inoculating Vero cells with a low MOI infection.
Virus-containing supernatant was harvested when the cells exhibited more than 80\% cytopathic effect and stored at -80 \degree C.
We quantified viral titers of virus stock using TCID-50 assays, calculated by the Spearman-Karber method \citep{shao2016}.
When mixed 1:1 with the red blood cell mixture, the final concentration of virus in the blood meal was 3.540 x $10^6$ $TCID_{50}$/mL.

Adult mosquitoes were aggregated by site and stored in reach-in incubators at $27 \degree C \pm 0.5 \degree C$, $80\% \pm 5\%$ relative humidity, and a 12:12 hour light:dark cycle.
To ensure infected mosquitoes were of a similar age, mosquitoes were pooled into cohorts of 4-6 days old in the summer and 4-9 days old in the fall (due to slower and more asynchronous emergence rates), allowed to mate, and were fed \textit{ad libitum} with a 10\% sucrose solution.
Forty-eight hours prior to infection, the sucrose was replaced with deionized water, which was then removed 12-14 hours before infection.
Infectious blood meals were prepared as described in Shan et al. \citep{shan2016} and administered to mosquitoes through a water-jacketed membrane feeder.
Blood-fed mosquitoes were then maintained as described above for the duration of the experiment.

We assessed mosquitoes for infection (bodies positive for virus), dissemination (heads positive for virus), and infectiousness (saliva positive for virus) through dissections and salivation assays 21 days post infection following Tesla et al. \citep{tesla2017}.
To determine infection status, we used cytopathic effect (CPE) assays to test for the presence of virus in each collected tissue \citep{balaya1969}.
Individual bodies and heads were homogenized in 500 \si{\micro\liter} of DMEM and centrifuged at 2,500 rcf for 5 minutes. 200 \si{\micro\liter} of homogenate was added to Vero cells in a solution of DMEM (1\% pen-strep, 5\% FBS by volume) in a 24-well plate and kept at 37 \degree C and 5 \% ${CO_2}$.
Salivation media was thawed, and plated on Vero cells as above.
After 5 days, Vero cells were assessed for presence of DENV-2 via CPE assays.
Samples were identified as positive for virus if CPE was present in the well.

\subsection*{Intrinsic growth rates (r') and vectorial capacity (VC)}

We calculated the per capita population growth rate per following Livdahl and Sugihara \citep{livdahl1984} {Eq. \ref{eq:1}}:

\begin{equation} \label{eq:1}
r' = \frac{ln(\frac{1}{N_0}\sum_{x}^{ }{A_x}f(\bar{w_x}))}{D+\frac{\sum_{x}^{ }xA_xf(\bar{w_x})}{\sum_{x}^{ }A_xf(\bar{w_x})}}
\end{equation}

Where $N_0$ is the initial number of female mosquitoes (assumed to be 50\% of the larvae, n=50), $A_x$ is the number of mosquitoes emerging on day $x$, $D$ is the time to reproduction following emergence (assumed to be 14 days \citep{livdahl1991}), and $f(\bar{w_x})$ is fecundity as a function of mean wing size on day $x$ ($w_x$; Equation \ref{eq:2}).
This relationship is assumed to be linear and calculated via Lounibos et al. \citep{lounibos2002}:

\begin{equation} \label{eq:2}
f(\bar{w_x}) = -121.240 + (78.02 \times \bar{w_x})
\end{equation}

We calculated the vectorial capacity ($VC$; Equation \ref{eq:3}) for each site and season using a temperature-dependent mechanistic dengue model defined in Mordecai et al. \citep{mordecai2017}.

\begin{equation} \label{eq:3}
VC(T) =\frac{a(T)^2b(T)c(T)e^{-\mu (T)/EIR(T)} EFD(T) p_{EA}(T) MDR(T)} {\mu(T)^2}
\end{equation}

Here, mosquito traits are a function of temperature, $T$, as described in Table 1:

\begin{table}[h]
\begin{flushleft}
\textbf{Table 1. Sources of parameters used in the $VC$ equation.}
\end{flushleft}
\begin{tabular}{|P{0.15\linewidth}P{0.3\linewidth}P{0.25\linewidth}P{0.25\linewidth}|}
\hline
\textbf{Parameter} & \textbf{Definition} & \textbf{Without carry-over effects} & \textbf{With carry-over effects}\\
\hline
$a(T)$ & Per-mosquito bite rate & Mordecai et al. 2017 & Mordecai et al. 2017 \\
$b(T)c(T)$* & Vector competence & Mordecai et al. 2017 & Current Study \\
$\mu(T)$ & Adult mosquito mortality rate & Mordecai et al. 2017 & Mordecai et al. 2017 \\
$EIR(T)$ & Extrinsic incubation rate (inverse of extrinsic incubation period) & Mordecai et al. 2017 & Mordecai et al. 2017 \\
$EFD(T)$* & Number of eggs produced per female mosquito per day & Mordecai et al. 2017 & Current Study \\
$p_{EA}(T)$ & Egg-to-adult survival probability & Current Study & Current Study \\
$MDR(T)$ & Mosquito immature development rate & Current Study & Current Study \\
\hline
\end{tabular}
\\[1.5pt]

Parameters sourced from \citep{mordecai2017} were mathematically estimated at a constant temperature of 27 \degree C. Parameters that included carry-over effects are starred.
\end{table}

Site-level $VC$ was calculated using a combination of traits empirically measured in this study and traits estimated from thermal response models as described in \citep{mordecai2017}.
The bite rate ($a(T)$), adult mosquito mortality rate ($\mu(T)$), and extrinsic incubation rate ($EIR(T)$), were calculated for mosquitoes at a constant 27 \degree C using temperature dependent functions from \citep{mordecai2017}.
Vector competence ($b(T)c(T)$) was calculated as the proportion of infectious mosquitoes per site as found by our dengue infection assays.
The number of eggs produced per female per day ($EFD(T)$) was calculated by estimating fecundity from average female wing length following Eq. \ref{eq:2}, and then dividing this by the expected lifespan of mosquitoes ($1/\mu$).
The egg-to-adult survival probability ($p_{EA}(T)$) was defined as the average proportion of adults emerging at a site.
The mosquito immature development rate ($MDR(T)$) was calculated as the inverse of the mean time to emergence for female mosquitoes per site, resulting in a daily rate of development.

To estimate vectorial capacity with and without carry-over effects, we constructed two models.
The model without carry-over effects used mathematically estimated values for vector competence and fecundity based on thermal response models calculated at the adult environmental temperature (27 \degree C) following Murdock et al. \citep{mordecai2017}, while the model incorporating carry-over effects used the empirically estimated values from our study.
All other parameters were the same across the two models.

\subsection*{Statistical Analysis}

All analyses were conducted with respect to the female subset of the population, as they are the subpopulation responsible for disease transmission.
In the case of data logger failure (N = 3), imputed means from the site were used to replace microclimate data.
In the case of trays failing due to wildlife tampering (two urban and one suburban in the fall replicate), collected mosquitoes were used for infection assays, but were excluded from survival and emergence analyses.
For all mixed-models, significance was assessed through Wald Chi-square tests ($\alpha=0.05$) and examination of 95\% confidence intervals.
Pearson residuals and Q-Q plots were visually inspected for normality.
All mixed models were fit using the \texttt{lme4} package in $R$.

We used generalized linear mixed models (GZLMs) to explore if microclimate (i.e. mean, minimum, maximum, and daily ranges of temperature and relative humidity), the mean proportion of adult females emerging per tray, time to female emergence, female body size, per capita growth rate, metrics of vector competence, and vectorial capacity differed across land class and season.
In all models, fixed effects included land class, season, and their interaction, and site was a random effect.
The effect of body size on infection dynamics was also explored at the level of the individual mosquito, fitting a binomial GZLM including wing size as a fixed effect and site as a random effect.

To explore whether observed effects of land class and season were due to variation in microclimate, we assessed the effects of microclimate on each response variable.
Due to extreme correlation between variables ($\rho>0.9$), we ultimately chose one variable to represent microclimate (mean temperature) to reduce bias due to collinearity \citep{graham2003}.
Thus, we fit GZLMs to each response variable described above with mean daily temperature and site as fixed and random effects, respectively.

\section*{Results}

\subsection*{Effects of land class and season on microclimate}

We found that microclimate profiles differed significantly across both season and land class (S4 Table, S2 Fig.). All microclimate metrics differed significantly across season, except for maximum relative humidity (\textit{z}=0.679, $p=0.497$). In general, temperatures were warmer in the summer and on urban sites, replicating what was found in a prior study in this system \citep{murdock2017}. Relative humidity was higher in the summer than the fall, due to a drought, and lower on urban land classes than suburban and rural classes.

\subsection*{Direct and carry-over effects of land class, season, and microclimate on population growth}

The total proportion of adult females emerging per tray was significantly higher in summer than fall (Table 2), but did not differ across land class (Fig. \ref{Fig:1}A). Of the 3,600 first-instar larvae placed in each season, a total of 2595 and 1128 mosquitoes emerged in the summer and fall, respectively. There was a strong positive relationship between mean daily temperature and larval survival to emergence by tray (Table 2). The mean rate of larval development per tray was significantly different between summer and fall (Fig. \ref{Fig:1}B, Table 2), with daily development rates of 0.074 $\pm$ 0.002 day\textsuperscript{-1} and 0.0387 $\pm$ 0.002 day\textsuperscript{-1}, respectively. There was a significant positive relationship between temperature and larval development rate (Table 2).

We did not observe a significant carry-over effect of land class or season on mosquito wing size, however there was a significant interaction between the two (Table 2), with the smallest mosquitoes on suburban and rural sites in the summer and fall, respectively. We also found no effect of temperature on female wing size. After incorporating the number of adult females emerging per day, the date of emergence, and their body size into the per capita growth rate equation (Eq. \ref{eq:1}), we found that the estimated per capita growth rate was higher in the summer season than the fall season (Fig. \ref{Fig:1}C, Table 2). There was no evidence for a difference in population growth across land class or temperature.

\subsection*{Carry-over effects of land class, season, and microclimate on vector competence}

A total of 319 female mosquitoes were assessed for infection status, 20 per site in the summer and varying numbers per site in the fall due to lower emergence rates (sample sizes reported in S5 Table). We found that land class and season did significantly impact the probability of a mosquito becoming infected and disseminating dengue infection (Fig. \ref{Fig:1}D, E), Table 2). The probability of becoming infectious did not differ across land class, nor season (Fig. \ref{Fig:1}F), despite the higher probability of mosquito infection and dissemination in the fall and on suburban and rural sites. This suggests that the ability of virus to penetrate the salivary glands differs in adults reared in the summer vs. the fall and across land class, with a higher proportion of dengue infected mosquitoes becoming infectious in the summer and on urban sites (S5 Table, $\chi^2=13.65$, $p<0.001$). We also found the probability of infection to decline with increasing body size ($\chi^2=4.776$, $p=0.0289$), although there was no evidence for a relationship between body size and the probability of dissemination or infectiousness. Differences in infection status across land class and season were driven by a strong relationship with microclimate. We found that infection and dissemination rates decreased with increasing temperatures, while there was no relationship between infectiousness and temperature (Table 2).

\subsection*{Integrating direct and carry-over effects into estimates of transmission potential}

When calculating $VC$ with or without the inclusion of carry-over effects, $VC$ was higher in the summer than the fall (S3 Fig., Table 2). In the summer season, there was a trend for $VC$ to increase with increasing urbanization (S3 Fig.). This trend was not significant, however, given the small sample size (n=9) and the disproportional impact of having no infectious mosquitoes at one site, resulting in a value of $VC=0$ for one sample. Further, we found that calculated vectorial capacity increased with temperature for both models, although the increase was more pronounced when not accounting for carry-over effects (Fig. \ref{Fig:VC}). When comparing $VC$ calculations with and without carry-over effects, we found that including carry-over effects decreased the expected vectorial capacity overall by an average of 84.89 $\pm$ 2.86 \% (S3 Fig.).

\section*{Discussion}

Mathematical models of mosquito-borne disease rarely include mosquito larval stages \citep{reiner2013}, and of those that do, few include the influence of carry-over effects on important mosquito life-history traits (but see \citep{roux2015a}). This is likely because there are relatively few empirical studies parameterizing carry-over effects in mosquito-pathogen systems \citep{parham2015}, and, most are laboratory studies conducted across a wider range of temperatures than that seen in the field. Here, we demonstrate that fine-scale differences in larval microclimate generate carry-over effects on adult vector competence and fecundity, resulting in variation in predicted mosquito population dynamics and transmission potential across an urban landscape and season.

The subtle heterogeneity in microclimate we observed resulted in significantly different predicted population growth rates through its effects on demographic traits. Daily mean temperatures (25.43 \degree C) across all sites in the summer were closer to the predicted thermal optimum of \textit{Ae. albopictus} for the probability of egg to adult survival (24-25 \degree C) \citep{mordecai2017} than in the fall (17.69 \degree C), leading to higher survival rates. We also observed more rapid larval development rates in the summer relative to the fall, and on warmer urban sites in the fall only. Again, this is likely due to the strong positive relationship observed between development rates and  mean larval temperature, as the metabolic rate of mosquitoes will increase with warming temperatures \citep{delatte2009}.

Surprisingly, we found no effect of land class or season on female mosquito body size, despite the difference in temperatures across season. Following allometric temperature-size relationships of ectotherms, warmer larval temperatures should lead to smaller bodied mosquitoes \citep{angilleta2004}. However, our results contrast with many laboratory and field studies that have found a negative relationship between rearing temperature and mosquito body size (\textit{Ae. albopictus} \citep{reiskind2012a, murdock2017}, \textit{Culex tarsalis} \citep{dodson2012}, \textit{Anopheles gambiae} \citep{koella1996}). This may be due to a difference in nutrient quality. Nutrient availability and quality can mediate the relationship between temperature and body size \citep{farjana2011}. The majority of the above laboratory studies rear larvae on high quality food sources, such as fish food. The leaf infusion used in our experiment relied on yeast and naturally colonizing microorganisms that grow more slowly at low temperatures \citep{ratkowsky1982}, likely constraining larval growth. For example, Lounibos et al. \citep{lounibos2002} found a positive relationship between temperature and male \textit{Ae. albopictus} body size when larvae were provided leaf litter.

Our results agree with laboratory studies in other arboviral systems (chikungunya \citep{adelman2013}, yellow fever \citep{adelman2013}, and Rift Valley fever \citep{turell1993}) that found cool larval environmental temperatures to enhance arbovirus infection relative to warmer larval environments. Studies in the \textit{Ae. albopictus}-dengue virus system have also found that low larval temperatures enhance mosquito susceptibility to viral infection, although this is dependent on larval nutrition \citep{buckner2016} and the stage of the infection (i.e. mid-gut vs. dissemination vs. saliva) \citep{alto2013}. While we found infection and dissemination to decrease with increasing temperatures, there was no effect of temperature on viral presence in the saliva, suggesting carry over effects due to microclimate variation may alter the overall efficiency of dengue infection. Thus, even though a smaller proportion of mosquitoes reared on urban sites and in the summer became infected and disseminated infection, these mosquitoes were more likely to become infectious, resulting in no net difference in overall vector competence across land class and season. Thus, later stages of viral infection (i.e. salivary gland penetration) may be differentially impacted by larval environmental temperature than earlier stages (i.e. midgut escape).

Current models of vector-borne disease focus primarily on direct effects of environmental variables on mosquito densities and disease transmission and rarely include the effects of the larval stage, either directly or via carry-over effects \citep{reiner2013}. We find that when carry-over effects are not incorporated, mechanistic models overestimate the effects of key environmental drivers (e.g. temperature) on transmission. The relatively small differences in temperature across our study site (less than 1.5 \degree C) resulted in a two-fold difference in predicted vectorial capacity when omitting carry-over effects. Thus, we would expect these phenomena to have an even larger impact in more urbanized areas, particularly megacities, with larger seasonal and spatial microclimate ranges \citep{peng2012}.

Carry-over effects are not simply limited to microclimate, and can result due to variation in larval nutrition \citep{moller-jacobs2014}, intra- and inter-specific densities \citep{alto2005}, and predation \citep{roux2015a} in mosquito systems. Further, abiotic and biotic factors will likely interact to influence carry over effects \citep{buckner2016, muturi2012a}, and this interaction could be scale-dependent \citep{leisnham2014}. For example, biotic processes are predicted to be more important at local geographic scales, while abiotic processes dominate at regional geographic scales in species distribution models \citep{cohen2016}. Future exploration of the scale-dependent contribution of different environmental factors and their interactive influence on both direct and carry-over effects is needed to improve models of mosquito distributions, population dynamics, and disease transmission.

In conclusion, we found fine-scale variation in microclimate to shape mosquito population dynamics and arbovirus transmission potential through direct effects on larval survival and development rates, and indirectly through carry-over effects on vector competence and fecundity. Given the devastating impact of disease in other species with complex life histories (e.g. chytridiomycosis in amphibians), the role of carry-over effects in disease transmission are an important, though understudied, mechanism that must be better understood to control disease spread. Thus, incorporating relationships between carry-over effects and organismal life-history traits into statistical and mechanistic models will lead to more accurate predictions on the distributions of species, population dynamics, and the transmission of pathogens and parasites. The interaction between the larval and adult environments, mediated by carry-over effects, could have complex consequences for adult phenotypes and fitness for mosquitoes as well as other organisms.

%%%%%%%%%%%%%%%%%%%%%%%%%%%%%%%%%%%%%%%%%%%%%%
%%                                          %%
%% Backmatter begins here                   %%
%%                                          %%
%%%%%%%%%%%%%%%%%%%%%%%%%%%%%%%%%%%%%%%%%%%%%%

\begin{backmatter}

\section*{Competing interests}
  The authors declare that they have no competing interests.

\section*{Author's contributions}
    Text for this section \ldots

\section*{Acknowledgements}
We thank members of the Murdock and Brindley labs for discussion and technical support conducting viral assays. We thank Diana Diaz, Abigail Lecroy, and Marco Notarangelo for assistance in the field and lab.
%%%%%%%%%%%%%%%%%%%%%%%%%%%%%%%%%%%%%%%%%%%%%%%%%%%%%%%%%%%%%
%%                  The Bibliography                       %%
%%                                                         %%
%%  Bmc_mathpys.bst  will be used to                       %%
%%  create a .BBL file for submission.                     %%
%%  After submission of the .TEX file,                     %%
%%  you will be prompted to submit your .BBL file.         %%
%%                                                         %%
%%                                                         %%
%%  Note that the displayed Bibliography will not          %%
%%  necessarily be rendered by Latex exactly as specified  %%
%%  in the online Instructions for Authors.                %%
%%                                                         %%
%%%%%%%%%%%%%%%%%%%%%%%%%%%%%%%%%%%%%%%%%%%%%%%%%%%%%%%%%%%%%

% if your bibliography is in bibtex format, use those commands:
\bibliographystyle{bmc-mathphys} % Style BST file (bmc-mathphys, vancouver, spbasic).
\bibliography{bmc_article}      % Bibliography file (usually '*.bib' )
% for author-year bibliography (bmc-mathphys or spbasic)
% a) write to bib file (bmc-mathphys only)
% @settings{label, options="nameyear"}
% b) uncomment next line
%\nocite{label}

% or include bibliography directly:
% \begin{thebibliography}
% \bibitem{b1}
% \end{thebibliography}

%%%%%%%%%%%%%%%%%%%%%%%%%%%%%%%%%%%
%%                               %%
%% Figures                       %%
%%                               %%
%% NB: this is for captions and  %%
%% Titles. All graphics must be  %%
%% submitted separately and NOT  %%
%% included in the Tex document  %%
%%                               %%
%%%%%%%%%%%%%%%%%%%%%%%%%%%%%%%%%%%

%%
%% Do not use \listoffigures as most will included as separate files

\section*{Figures}

\begin{figure}[h!]
  \label{Fig:map}
  \caption{\csentence{Map of study sites in Athens, GA.}
      Inset illustrates location of Athens-Clarke County (black outline) in the state of Georgia. Symbols represent land classes (square: rural, circle:suburban, triangle: urban). Colors represent the amount of impervious surface within the 210m focal area of each pixel, as illustrated on the color bar on the bottom.}
\end{figure}

\begin{figure}[h!]
  \label{Fig:climate}
  \caption{\csentence{Temperature and relative humidity across season and land class.}
      The solid line represents the mean temperature and relative humidity across trays in each land class. The dotted lines represent the mean minimum and maximum temperature and relative humidity across trays in each land class.}
\end{figure}

\begin{figure}[h!]
  \caption{\csentence{Sample figure title.}
      Figure legend text.}
\end{figure}

%%%%%%%%%%%%%%%%%%%%%%%%%%%%%%%%%%%
%%                               %%
%% Tables                        %%
%%                               %%
%%%%%%%%%%%%%%%%%%%%%%%%%%%%%%%%%%%

%% Use of \listoftables is discouraged.
%%
\section*{Tables}
\begin{table}[h!]
\caption{Sample table title. This is where the description of the table should go.}
      \begin{tabular}{cccc}
        \hline
           & B1  &B2   & B3\\ \hline
        A1 & 0.1 & 0.2 & 0.3\\
        A2 & ... & ..  & .\\
        A3 & ..  & .   & .\\ \hline
      \end{tabular}
\end{table}

%%%%%%%%%%%%%%%%%%%%%%%%%%%%%%%%%%%
%%                               %%
%% Additional Files              %%
%%                               %%
%%%%%%%%%%%%%%%%%%%%%%%%%%%%%%%%%%%

\section*{Additional Files}
  \subsection*{Additional file 1 --- Sample additional file title}
    Additional file descriptions text (including details of how to
    view the file, if it is in a non-standard format or the file extension).  This might
    refer to a multi-page table or a figure.

  \subsection*{Additional file 2 --- Sample additional file title}
    Additional file descriptions text.


\end{backmatter}
\end{document}
