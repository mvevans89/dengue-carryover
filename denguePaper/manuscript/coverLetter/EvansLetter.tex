\documentclass{letter}
\usepackage[english]{babel}
\usepackage{graphicx}
\addtolength{\voffset}{-2.95in}
\addtolength{\hoffset}{-0.3in}
\addtolength{\textheight}{12cm}


\begin{document}
%--------------------
%	ADDRESSEE
%--------------------

\begin{letter}

%-------------------
%  UGA banner image
%-------------------

\vspace{-2in}
\includegraphics[width=\textwidth]{thin4c.png}

%------date
\begin{flushright}
\today
\end{flushright}

%-----------------manuscript in reference

Re: Carry-over effects of larval microclimate on the transmission potential of a mosquito-borne pathogen\\

%-----------------------------------------------------------
Dear Editor:\\

Please find enclosed with this letter our manuscript ``Carry-over effects of larval microclimate on the transmission potential of a mosquito-borne pathogen" which we submit for publication as a Research Article in \textit{{Proceedings B}}.\\

%%-----This paragraph is especially key. Should be strong, accurate and concise, Typically three points. Includes summary, rationale for publication, generalization or intended audience.
This manuscript reports the results of a semi-field experiment exploring the influence of cross life-stage carry-over effects on mosquito-borne disease. Given its strong influence on mosquito life history, temperature is a strong predictor of mosquito-borne disease in mechanistic models of disease transmission. However, the net-effect of field-based environmental variation in the larval environment on adult life history traits relevant to disease transmission in adults is unknown. Here, we use data collected from a semi-field experiment investigating dengue dynamics in \textit{Aedes albopictus} mosquitoes across relevant environmental variation to parameterize a mechanistic dengue transmission model, allowing comparison to models that omit carry-over effects of the larval environment. We found that the larval environment significantly influenced mosquito susceptibility to dengue virus, and inclusion of these effects in a mechanistic model weakened the relationship between temperature and transmission potential considerably. Given the growing use of mechanistic models in the prediction of mosquito-borne disease risk to human populations, determining which aspects of mosquito ecology are relevant to disease transmission is crucial to improving model accuracy. Our findings identify carry-over effects of the larval environment as one such phenomena that can drastically alter predictions of disease transmission potential, and warrants further investigation. \\

I can be reached by email (mvevans@uga.edu), phone (703 725 9580), or post (Odum School of Ecology, University of Georgia, Athens GA 30602). Thank you for your consideration. \\

Sincerely,\\
Michelle Evans

\vspace{1cm}

\footnotesize
%assertion of ethics
We confirm that this manuscript has not been published, is not in press, and is not under review elsewhere. All data and \texttt{R} code needed to reproduce the analysis and plotting will be deposited on Figshare upon acceptance.



\end{letter}
\end{document}
