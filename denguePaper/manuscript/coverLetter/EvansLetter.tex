\documentclass{letter}
\usepackage[english]{babel}
\usepackage{graphicx}
\addtolength{\voffset}{-2.95in}
\addtolength{\hoffset}{-0.3in}
\addtolength{\textheight}{12cm}


\begin{document}
%--------------------
%	ADDRESSEE
%--------------------

\begin{letter}

%-------------------
%  UGA banner image
%-------------------

\vspace{-2in}
\includegraphics[width=\textwidth]{thin4c.png}

%------date
\begin{flushright}
\today
\end{flushright}

%-----------------manuscript in reference

Re: Carry-over effects of urban larval environment on the transmission potential of a mosquito-borne pathogen\\

%-----------------------------------------------------------
Dear Editor:\\

Please find enclosed with this letter our manuscript ``Carry-over effects of urban larval environment on the transmission potential of a mosquito-borne pathogen" which we submit for publication as a Research Article in \textit{Parasites and Vectors}.\\

%%-----This paragraph is especially key. Should be strong, accurate and concise, Typically three points. Includes summary, rationale for publication, generalization or intended audience.
This manuscript reports the results of a semi-field experiment exploring the influence of cross life-stage carry-over effects on mosquito-borne disease. Given its strong influence on mosquito life history, temperature is a strong predictor of mosquito-borne disease in mechanistic models of disease transmission. However, temperature fluctuates across space and time, particularly in urban environemnts, and the net-effect of urbanization on carry-over effects, particularly vector competence, is unknown. Here, we conduct a semi-field experiment exploring population and dengue virus 2 (DENV-2) transmission relevant life-history traits from \textit{Aedes albopictus} mosquitoes reared in three urban land classes across the summer and fall. We find that the larval environment significantly influences mosquito susceptibility to dengue virus, with mosquitoes reared in warmer summer environments more competent for the disease. Ommitting land-class and season specific carry-over effects leads to an underestimate of disease potential on urban sites in the summer. Given the growing use of mechanistic models in the prediction of mosquito-borne disease risk to human populations, determining which aspects of mosquito ecology are relevant to disease transmission is crucial to improving model accuracy. Our findings identify carry-over effects of the larval environment as one such phenomena that can drastically alter predictions of disease transmission potential, and warrants further investigation. \\

I can be reached by email (mvevans@uga.edu), phone (703 725 9580), or post (Odum School of Ecology, University of Georgia, Athens GA 30602). Thank you for your consideration. \\

Sincerely,\\
Michelle Evans

\vspace{1cm}

\footnotesize
%assertion of ethics
We confirm that this manuscript has not been published, is not in press, and is not under review elsewhere. All data and \texttt{R} code needed to reproduce the analysis and plotting will be deposited on Figshare upon acceptance.



\end{letter}
\end{document}
