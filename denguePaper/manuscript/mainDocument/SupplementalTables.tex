\documentclass[12pt]{article}
\usepackage[paper=letterpaper, margin=0.8in]{geometry}

%% The graphicx package provides the includegraphics command.
\usepackage{graphicx}
%% The amssymb package provides various useful mathematical symbols
\usepackage{amssymb}
%% The amsthm package provides extended theorem environments
%% \usepackage{amsthm}
%% fix strange gensymb error
\usepackage{textcomp}
%% symbols, especially degree
\usepackage{gensymb}
%% scientific units
\usepackage{siunitx}
%% line spacing
\usepackage{setspace}
\doublespacing
%% put figures at the end
%\usepackage[nomarkers]{endfloat}
%% allow hyperlinks
%\usepackage{hyperref}
%% color comments
\usepackage{soul}
\usepackage{color}
%% left justification in tables
\usepackage{array}
\newcolumntype{P}[1]{>{\raggedright\arraybackslash}p{#1}}
%%references
\usepackage[round]{natbib}
%%landscape orientation
\usepackage{pdflscape}


\begin{document}

{\Large
\textbf\newline{Carry-over effects of larval microclimate on the transmission potential of a mosquito-borne pathogen}}

\bigskip

Michelle V. Evans,
Justine C. Shiau,
Nicole Solano,
Melinda A. Brindley,
John M. Drake,
Courtney C. Murdock
\smallskip

{\Large{Supplemental Tables}}

\newpage

\begin{landscape}

\begin{table}[]
\centering
\caption{\textbf{Supplemental Table 1.} Mean $\pm$ s.e. of climate metrics across season and land class. Superscripts represent significant differences within a season as measured by pair-wise comparisons using Tukey multiple comparisons of means, adjusting for significance with the Holm-Bonferroni method.}
\begin{tabular}{l|ccc|ccc}
	                       & \multicolumn{3}{c}{Summer}                            & \multicolumn{3}{c}{Fall}                              \\
	\hline
	                       & Rural            & Suburban        & Urban            & Rural            & Suburban         & Urban           \\
	\hline
	Min. Temperature       & $21.69 \pm 0.23^a$  & $21.99 \pm 0.32^a$ & $22.69 \pm 0.33^a$  & $11.07 \pm 0.33^a$  & $12.23 \pm 0.47^{ab}$ & $13.32 \pm 0.47^b$ \\
	Mean Temperature       & $25.41 \pm 0.16^a$  & $25.36 \pm 0.22^a$ & $26.30 \pm 0.23^b$  & $17.87 \pm 0.23^a$  & $18.11 \pm 0.33^a$  & $19.28 \pm 0.33^b$ \\
	Max. Temperature       & $31.39 \pm 0.49^a$  & $30.86 \pm 0.68^a$ & $31.41 \pm 0.71^a$  & $27.52 \pm 0.37^a$  & $26.58 \pm 0.53^a$  & $26.87 \pm 0.55^a$ \\
	DTR                    & $9.82 \pm 0.55^a$   & $8.86 \pm 0.76^a$  & $8.75 \pm 0.79^a$   & $16.46 \pm 0.57^a$  & $14.35 \pm 0.80^a$  & $13.58 \pm 0.82^b$ \\
	Min. Relative Humidity & $73.59 \pm 1.84^{ab}$ & $76.29 \pm 2.55^b$ & $66.48 \pm 2.67^a$  & $47.81 \pm 1.75^a$  & $48.84 \pm 2.45^a$  & $44.28 \pm 2.52^a$ \\
	Mean Relative Humidity & $93.80 \pm 1.04^a$ & $94.77 \pm 1.45^a$ & $87.52 \pm 1.51^b$ & $80.84  \pm 1.52^a$ & $80.41 \pm 2.14^a$  & $71.58 \pm 2.17^b$ \\
	Max. Relative Humidity & $99.97 \pm 0.84^a$  & $99.97 \pm 1.31^a$ & $98.09 \pm 1.37^a$  & $99.33 \pm 1.25^a$  & $98.92 \pm 1.75^a$  & $92.07 \pm 1.78^b$ \\
	DHR                    & $26.37 \pm 2.00^{ab}$ & $23.69 \pm 2.77^b$ & $31.62 \pm 2.90^a$  & $51.51 \pm 1.89^a$  & $50.09 \pm 2.65^a$  & $47.79 \pm 2.73^a$
\end{tabular}
\end{table}

\clearpage

\begin{table}[h]
\centering
\begin{tabular}{P{0.12\linewidth}P{0.12\linewidth}P{0.12\linewidth}P{0.12\linewidth}P{0.15\linewidth}P{0.15\linewidth}}
\hline
\textbf{Season} & \textbf{Land Class} & \textbf{No. tested} & \textbf{No. infected (\%)}  & \textbf{No. disseminated (\%)} & \textbf{No. infectious (\%)} \\
\hline
\hline
\textit{Summer} &   &   &   &   \\
     & Rural & 56 & 22 (39) & 19 (48) & 6 (15) \\
   & Suburban & 57 & 32 (56) & 26 (81) & 10 (31) \\
   & Urban & 51 & 10 (20) & 10 (100) & 7 (70) \\

\textit{Fall} &   &   &   &   \\
     & Rural & 50 & 32 (64) & 30 (94) & 3 (9) \\
   & Suburban & 43 & 28 (65) & 25 (89) & 3 (11) \\
   & Urban & 51 & 10 (20) & 10 (100) & 7 (70) \\
\hline
\end{tabular}
\caption{\textbf{Supplemental Table 2.} The efficiency rates of infection (mosquitoes with dengue positive bodies), dissemination (infected mosquitoes with dengue positive heads) and infectiousness (infected mosquitoes with dengue positive saliva) across season and land class. Raw numbers of positive samples are shown with percentages in parentheses.}
\end{table}

\end{landscape}

\end{document}
